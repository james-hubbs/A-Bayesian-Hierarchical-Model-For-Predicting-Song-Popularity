\usepackage{amsmath, amssymb, amsthm}
\usepackage{algorithm}
\usepackage{algpseudocode}
\usepackage{amsmath, amssymb, amsthm}
\usepackage{float, fullpage, graphicx, multirow,parskip, subcaption, setspace}
\usepackage{comment}
\usepackage{url}
\usepackage{enumitem}
\usepackage{tikz}
\usepackage{bm, bbm}
\usetikzlibrary{shapes.geometric, positioning}
\usetikzlibrary{quotes, angles}
\usepackage{rotating}
\usepackage{hyperref}
\usepackage{natbib}


% Define some colors
\definecolor{SkyBlue}{RGB}{14, 118, 188}
\definecolor{BrightRed}{RGB}{223, 82, 78}
\definecolor{Green638}{RGB}{165,255,118} % from colours.cafe on instagram; pallete638

% Set up colorful hyperlinks without any silly green boxes
\hypersetup{pdfborder = {0 0 0.5 [3 3]}, colorlinks = true, linkcolor = BrightRed, citecolor = SkyBlue}

\bibliographystyle{apalike}

% comment
\newcommand{\skd}[1]{\textcolor{cyan}{\small [skd]: #1}}


% Math macros
\DeclareMathOperator*{\argmax}{arg\,max}
\DeclareMathOperator*{\argmin}{arg\,min}

\newcommand\numberthis{\addtocounter{equation}{1}\tag{\theequation}} % useful if we want to number one equation inside an align*
\newcommand\numbereqn{\addtocounter{equation}{1}\tag{\theequation}}

\newcommand{\R}{\mathbb{R}} % boldfaced R for the reals
\newcommand{\E}{\mathbb{E}} % boldfaced E for expectations
\def\P{\mathbb{P}} % boldfaced P for probability. overriding \P for paragraph symbol

\newcommand{\calP}{\mathcal{P}} % caligraphic P for a generic distribution
\newcommand{\calQ}{\mathcal{Q}} % caligraphic Q for another generic distribution
\newcommand{\calF}{\mathcal{F}} % caligraphic F, typically for sigma-algebras
\newcommand{\calG}{\mathcal{G}}
\newcommand{\calX}{\mathcal{X}}
\newcommand{\calC}{\mathcal{C}}

\newcommand{\ind}[1]{\mathbbm{1}\left( #1 \right)} % indicator function, with an argument
\newcommand{\var}[1]{\textrm{Var}\left( #1 \right)} % variance
\newcommand{\cov}[2]{\textrm{Cov}\left( #1, #2 \right)} % covariance
\newcommand{\sign}[1]{\textrm{sign}\left(#1\right)} % sign
\newcommand{\parallelsum}{\mathbin{\|}} % for double bar to behave like a binary operation
\newcommand{\kl}[2]{\textrm{KL}\left(#1 \mid \parallelsum \# \right)} % KL divergence with two arguments


% distributions
\newcommand{\normaldist}[2]{\mathcal{N}\left(#1,~#2\right)} % normal distribution
\newcommand{\mvnormaldist}[3]{\mathcal{N}_{#1}\left(#2,~#3\right)} % multivariate normal distribution
\newcommand{\gammadist}[2]{\textrm{Gamma}\left(#1,~#2\right)} % gamma distribution
\newcommand{\igammadist}[2]{\textrm{Inv.~Gamma}\left(#1,~#2\right)} % inverse gamma
\newcommand{\binomialdist}[2]{\textrm{Binomial}\left(#1,~#2\right)} % Binomial
\newcommand{\berndist}[1]{\textrm{Bernoulli}\left(#1\right)} % Bernoulli
\newcommand{\poisdist}[1]{\textrm{Poisson}\left(#1\right)} % Poisson
\newcommand{\hafltdist}[2]{\textrm{half-t}_{\#1}\left(#2\right)} %half-t
\newcommand{\unifdist}[2]{\textrm{Uniform}\left(#1,~#2\right)} % uniform
\newcommand{\betadist}[2]{\textrm{Beta}\left(#1,~#2\right)} % Beta distribution

% bolded alphabet time
\newcommand{\by}{\bm{y}}
\newcommand{\bx}{\bm{x}}
\newcommand{\bz}{\bm{z}}
\newcommand{\bw}{\bm{w}}
\newcommand{\br}{\bm{r}}
\def\bf{\bm{f}}
% specific for this work on trees
\newcommand{\pcont}{p_{\text{cont}}}
\newcommand{\pcat}{p_{\text{cat}}}


\newcommand{\Told}{T_{\text{old}}}
\newcommand{\Tnew}{T_{\text{new}}}


% bolded capitalized alphabet
\newcommand{\bY}{\bm{Y}}
\newcommand{\bX}{\bm{X}}

\newcommand{\nx}{\texttt{nx}}
\newcommand{\nxl}{\texttt{nxl}}
\newcommand{\nxr}{\texttt{nxr}}
\newcommand{\nxtil}{\tilde{\nx}}
\newcommand{\nleaf}{\texttt{nleaf}}
\newcommand{\nnog}{\texttt{nnog}}

\newcommand{\bphi}{\boldsymbol{\varphi}}

\newcommand{\Ecal}{\mathcal{E}}


% bolded greek alphabet time!
\newcommand{\btheta}{\boldsymbol{\theta}}
\newcommand{\bbeta}{\boldsymbol{\beta}}

%overline time
\newcommand{\ybar}{\overline{y}}
\newcommand{\xbar}{\overline{x}}
\newcommand{\mubar}{\overline{\mu}}


% ATT, etc
\newcommand{\ATT}[2]{\operatorname{ATT}\left(#1,#2\right)}



% Theorem-like declarations
\theoremstyle{plain}
\newtheorem{theorem}{Theorem}
\newtheorem{corollary}[theorem]{Corollary}
\newtheorem{lemma}[theorem]{Lemma}
\newtheorem{proposition}[theorem]{Proposition}

\theoremstyle{definition}
\newtheorem{definition}[theorem]{Definition}

\newtheorem{ex}{Example}
\newenvironment{example}{\begin{ex}}{ \hfill $\blacksquare$\end{ex}}


\theoremstyle{remark}
\newtheorem{remark}[theorem]{Remark}